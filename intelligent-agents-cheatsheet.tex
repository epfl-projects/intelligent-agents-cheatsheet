\documentclass[10pt,a4paper,landscape]{article}
\usepackage{multicol}
\usepackage{calc}
\usepackage{ifthen}
\usepackage[landscape]{geometry}
\usepackage{amsmath,amsthm,amsfonts,amssymb,mathtools}
\usepackage{color,graphicx}
\usepackage[dvipsnames]{xcolor}
\usepackage{hyperref}
\usepackage{listings}
\usepackage{underscore}
\usepackage{todonotes}

% Cheatsheet style
\input{style.tex}

% Shorthands
\renewcommand{\bf}[1]{\ensuremath{\mathbf{#1}}}
\newcommand{\E}{\mathrm{E}}
\newcommand{\Var}{\mathrm{Var}}
\newcommand{\Cov}{\mathrm{Cov}}
\newcommand{\balpha}{\boldsymbol\alpha}
\newcommand{\bbeta}{\boldsymbol\beta}
\newcommand{\bdelta}{\boldsymbol\delta}
\newcommand{\btheta}{\boldsymbol\theta}
\newcommand{\bPhi}{\boldsymbol\Phi}
\newcommand{\concept}[1]{\textcolor{Emerald}{#1}} % SeaGreen, SkyBlue, Emerald, Cerulean, CadetBlue
\newcommand{\subconcept}[1]{\textit{#1}}

\pdfinfo{
  /Title (Intelligent Agents Cheatsheet)
  /Creator (Merlin Nimier-David)
  /Author (Merlin Nimier-David)
  /Subject (Intelligent Agents cheatsheet)
  /Keywords (agent, artififical intelligence, planning, auctions, game theory)
}

% -----------------------------------------------------------------------

\begin{document}
\title{Intelligent Agents Cheatsheet}

\raggedright
\footnotesize
\sffamily
\begin{multicols*}{4}

% multicol parameters
% These lengths are set only within the two main columns
%\setlength{\columnseprule}{0.25pt}
\setlength{\premulticols}{1pt}
\setlength{\postmulticols}{1pt}
\setlength{\multicolsep}{1pt}
\setlength{\columnsep}{2pt}

\begin{center}
\Large{\underline{Intelligent Agents Cheatsheet}}
\end{center}

% ----------
\section{Reactive agents}

TODO: definition, stateless / stateful behavior, purely reactive decisions, future rewards, value / policy iteration algorithms, partial knowledge / state belief, Markov Decision Process.

% ----------
\section{Deliberative agents}

TODO: plannings, encoding of plans, utility, search algorithms (DFS, BFS, minimax, iterative deepening, Monte-Carlo search), notion of regret, multi-armed bandit.

TODO: factored representations, Bayesian networks, policy iteration with factored representation, operators (situation calculus), least commitment, planning as a SAT problem.

% ----------
\section{Multiagent systems}

TODO: different possible architectures \& types of allowed interactions, cooperative planning \& coordination, self-interested agents, partial-global planning, ontologies (heterogeneous agent systems), contract nets algorithm, market-based contract nets.

\subsection{Distributed multiagent systems}

TODO: social laws, marginal cost, planning as SAT (i.e. Constraint Satisfaction Problem), centralized / async backtracking, dynamic programming, distributed local search, breakout algorithm

% ----------
\section{Game theory}

TODO: self-interested agents, games (representations, normal form), zero-sum vs general-sum, dominant / pure / mixed / minimax strategies, (conditionally) dominated actions, Nash / Bayes-Nash / ex-post equilibria, utility theory, generalization to N-player games, games with uncertain utilities,

\subsection{Agent negotiation}

TODO: cooperation in games, correlated equilibria, mediators, negotiation protocols (rounds, time constraints, axiomatic negotiation, monotonic concession), Nash bargaining solution, generalization to N-agents games (mixed deals), truthfulness, private information.

\subsection{Mechanism design}

TODO: definition of mechanisms, Arrow's theorem (not really applicable), truthfulness, \textbf{Revelation principle}, \textbf{VCG tax} (implementation and its properties), generalized VCG mechanisms.

\subsection{Auctions}

TODO: goal: resource allocation, auction types \& properties (forward, reverse, open / sealed bids, first / second price): English, Dutch, Discriminatory, Vickery, Pareto efficiency, bidding strategies (truthful, optimal, risk).

TODO: vulnerability to collusion, multi-units / combinatorial  auctions, truthful information extraction, implementation considerations

TODO: double (two-way) auctions (clearing rules, incentive-compatibility), quotes, bargaining, sequential auctions (limitations), sunk-aware bidding, GVA (manipulable), procurement auctions, internet ads (GSP auctions).

\subsection{Coalitions}

\concept{Coalitions}: when \subconcept{side-contracts} (utility redistribution) are allowed, coalition utility $geq$ others $\implies$ stable.
\concept{Core} of a game: set of payoff distributions for which the grand coalition is stable (often empty).
Core is known to be nonempty for \subconcept{Superadditive} and \subconcept{Convex} games ($v(S \cup T) \geq v(S) + v(T) - v(S \cap T)$).
\concept{Shapley value}: unique vector (if core is nonempty, SV payoffs is in it). $SV(a_i) = $ average value of added payoff when added that agent to a sub-coalition (over all orders). Agents not in any carrier coalitions has $SV(a_i) = 0$.
\concept{Weighted graph games}: agents are odes, self-edges are payoffs, edges are payoffs for 2-coalitions (not possible for all games). Value of a coalition is the sum of edge weights in the subgraph.

\subsection{Voting protocols}

\concept{Manipulability}: non-truthful voting; removing a candidate can reverse the order; vote organizer can determine the winner by changing the order in which alternatives are presented.
\concept{Condorcet winner}: alternative that beats (or ties) all others in a pairwise majority vote (doesn't always exist; in a majority graph, it is the node with only outgoing edges).
\concept{Plurality voting}: vote for your single preferred alternative (variant: carry out $n - 1$ rounds, eliminate the least preferred at each round).
\concept{Borda count}: give $(n - 1) \dots 0$ points to alternatives.
\concept{Slater ranking}: vote between every pair of alternatives, pick the smallest transformation to obtain a majority graph.
\concept{Gibbard-Satterthwaite theorem}: any deterministic voting protocol ($\geq 3$ alternatives) has one of these properties: dictatorial, non-truthful, or some candidate cannot win.

% ---------- Page break
\newpage

% ---------- Other topics

% \section{Applications}
% \subsection{Agent-oriented software engineering}
% \subsection{Other gent applications}

% ---------- Credits
\section{Credits}
Most content taken from the lecture notes of Boi Falting's \href{http://edu.epfl.ch/coursebook/en/intelligent-agents-CS-430}{Intelligent Agents class} at EPFL, 2015.

% ---------- Footer
\vspace{0.5cm}
\hrule
\vspace{0.5cm}
\tiny
Rendered \today. Written by Merlin Nimier-David.
This work is licensed under the Creative Commons Attribution-ShareAlike 3.0 Unported License.
To view a copy of this license, visit \href{http://creativecommons.org/licenses/by-sa/3.0/}{http://creativecommons.org/licenses/by-sa/3.0/} or
send a letter to Creative Commons, 444 Castro Street, Suite 900, Mountain View, California, 94041, USA.
\end{multicols*}
\end{document}
